\chapter{Instruction Fetch}

The instruction fetcher of the core is able to supply one instruction to the ID
stage per cycle if the instruction cache or the instruction memory is able to
deliver an instruction after one cycle.
The instruction address must be word-aligned. It is not possible to jump to
misaligned memory addresses.

Branch prediction is used for branches where the branch decision is not yet
known, i.e. if the \instr{l.sf*} instruction precedes the \instr{l.bf} or
\instr{l.bnf} instruction directly.
Branch prediction assumes that backward branches are never taken and forward
branches are always taken. If the branch predicition guessed wrong, one fetched
instruction is wasted.

Table~\ref{tab:instr_signals} describes the signals that are used by to fetch
instructions.


\begin{table}[H]
 \caption{Instruction Fetch Signals}
 \label{tab:instr_signals}
  \begin{tabularx}{\textwidth}{@{}llX@{}} \toprule
    \textbf{Signal}                & \textbf{Direction} & \textbf{Description} \\ \toprule
    \signal{instr\_req\_o}         & \textbf{output}    & Request ready, must stay high until \signal{instr\_gnt\_i} is high for one cycle \\ \hline
    \signal{instr\_addr\_o[31:0]}  & \textbf{output}    & Address \\ \hline
    \signal{instr\_rdata\_i[31:0]} & \textbf{input}     & Data read from memory \\ \hline
    \signal{instr\_rvalid\_i}      & \textbf{input}     & \signal{instr\_rdata\_i} is valid now for this cycle. When \signal{instr\_rvalid\_i} is high, another request can be sent. \\ \hline
    \signal{instr\_gnt\_i}         & \textbf{input}     & The instruction cache accepted the request. The \signal{instr\_addr\_o} may be change in the next cylce. \\ \bottomrule
  \end{tabularx}
\end{table}


\section{Protocol}
The protocol used to communicate with the instruction cache or the instruction
memory is the same as the protocl used by the LSU. See the description of the
LSU in Section~\ref{sec:lsu_protocol} for details about the protocol.
