\chapter{Special-Purpose Registers}

\orion does not implement all special-purpose registers specified in the OR1K
standard, but is limited to the registers that were needed for the PULP system.
The reason for this is that we wanted to keep the footprint of the core as low
as possible and avoid any overhead that we do not explicitely need.

Figure~\ref{fig:spr_addr} shows the 16 bit SPR address format. In this section
only registers belonging to the group \texttt{SYS} (0) are described, see
Chapter~\ref{chap:perf_count} for the description of performance counter related
special-purpose registers. The description of hardware loop related
special-purpose registers is available in Chapter~\ref{chap:hwlp}

\begin{figure}[H]
  \centering
  \begin{bytefield}[endianness=big,bitwidth=15pt]{16}
    \bitheader{15,11,10,0} \\
    \bitbox{5}{  Group \#       } &
    \bitbox{11}{ Register \#    }\\
  \end{bytefield}
  \caption{SPR Address Format}
  \label{fig:spr_addr}
\end{figure}


An overview over the SPR addresses of the SYS group can be found in
Table~\ref{tab:sys_spr_addr}.

\begin{table}[H]
 \caption{SYS SPR Addresses}
 \label{tab:sys_spr_addr}
 \centering\begin{tabularx}{\textwidth}{@{}cclcX@{}} \toprule
   \textbf{Group \#} & \textbf{Reg \#} & Reg Name     & Access  & Description\\ \toprule
                   0 &  \texttt{0x010} & NPC          & R/W     & Next Program Counter \\ \hline
                   0 &  \texttt{0x011} & SR           & R/W     & Supervision Register \\ \hline
                   0 &  \texttt{0x012} & PPC          & R       & Previous Program Counter \\ \hline
                   0 &  \texttt{0x020} & EPCR         & R/W     & Exception Program Couner \\ \hline
                   0 &  \texttt{0x040} & ESR          & R/W     & Exception Supervision Register \\ \hline
                   0 &  \texttt{0x680} & Core ID      & R       & Core ID \\ \hline
                   0 &  \texttt{0x681} & Cluster ID   & R       & Cluster ID \\ \bottomrule
  \end{tabularx}
\end{table}


\section{Supervision Register (SR)}

\sprDesc{0x0011}{0x0000\_0000}{SR}{
  \begin{bytefield}[endianness=big,bitheight=60pt]{32}
    \bitheader{31,17,11,10,9,8,7,6,5,4,3,2,1,0} \\
    \bitbox{14}{ Unused                         }
    \bitbox{1}{\rotatebox{90}{\tiny PCOVE      }}
    \bitbox{5}{ Unused                         }
    \bitbox{1}{\rotatebox{90}{\tiny OV         }}
    \bitbox{1}{\rotatebox{90}{\tiny CY         }}
    \bitbox{1}{\rotatebox{90}{\tiny F          }}
    \bitbox{6}{ Unused                         }
    \bitbox{1}{\rotatebox{90}{\tiny IEE        }}
    \bitbox{2}{\rotatebox{90}{\tiny Unused     }}
  \end{bytefield}
}

\begin{itemize}
  \item[\textbf{PCOVE}] Performance Counter Overflow Enable \\
    When a performance counter register reaches its maximum value of
    \texttt{0xFFFFFFFF}, an interrupt will be triggered.

  \item[\textbf{CV}] Carry Bit\\
    This is the carry bit generated by the ALU

  \item[\textbf{CV}] Overflow Bit\\
    This is the overflow bit generated by the ALU

  \item[\textbf{F}] Flag \\
    The current state of the flag bit. This bit is usually set by a setflag
    instruction.

  \item[\textbf{IEE}] Interrupt Enable\\
    Enables interrupts for the core
\end{itemize}
