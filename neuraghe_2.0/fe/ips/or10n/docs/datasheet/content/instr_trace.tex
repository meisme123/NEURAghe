\chapter{Instruction Traces}
In the RTL simulation \orion is able to report traces of the instruction it has
been executing.

There are two flavors of those traces, a human readable version including
timestamps, cycle counter, program counter and the executed instruction in
hex-format, and a binary format that includes the current values of the
performance counters as well.

The binary format is intended for later post-processing by an external, for
example the pulp-pc-analyzer which can then be fed to KCacheGrind.
A description of the binary format is available below.


\section{Binary Format}

A binary instruction trace always starts with a header given in
Figure~\ref{fig:instr_trace_bin_header}. After the header multiple body elements
according to Figure~\ref{fig:instr_trace_el} follow.

\begin{figure}[H]
  \centering
  \begin{bytefield}[endianness=big]{32}
    \bitheader{31,0} \\
    \bitbox{32}{ Version                            } \\
    \bitbox{32}{ \#Metrics                          } \\
    \begin{rightwordgroup}{Metric 1}
      \bitbox{32}{ Absolute Number                    } \\
      \bitbox{32}{ Size                               } \\
      \bitbox{32}{ Delay                              } \\
      \wordbox{3}{ Description as string\\
                   Has an arbitrary byte size\\
                   Must terminate with a '\textbackslash0'}
    \end{rightwordgroup}\\
    \wordbox{2}{\vdots} \\
    \begin{rightwordgroup}{Metric N}
      \bitbox{32}{ Absolute Number                    } \\
      \bitbox{32}{ Size                               } \\
      \bitbox{32}{ Delay                              } \\
      \wordbox{3}{ Description as string\\
                   Has an arbitrary byte size\\
                   Must terminate with a '\textbackslash0'}
    \end{rightwordgroup}
  \end{bytefield}
  \caption{Instruction Trace: Binary Format Header.}
  \label{fig:instr_trace_bin_header}
\end{figure}

\begin{itemize}[leftmargin=2cm]
  \item[\textbf{Version}] Version number to be able to maintain backward compatibility.
    Current version is \texttt{0x00000001}

  \item[\textbf{\#Metrics}] Number of metrics

  \item[\textbf{Absolute Number}] Report difference to last number seen or absolute
    number to tool. A value of \texttt{1} means report absolute number directly.

  \item[\textbf{Size}] Size of metric in bytes. Currently only 4 is supported.

  \item[\textbf{Delay}] Report the number given in this metric for the current PC (0),
    the previous (-1) or the next (+1). This field is meant to offset reported
    metrics to the instruction that caused it. Since the pipeline of the CPU has
    a huge influence, it may happen that values reported in the binary trace are
    offset by a couple of cycles. \\
    E.g. an L2 access happens very late in the pipeline, but we report the
    instruction currently in the ID stage in the trace.
\end{itemize}

\begin{figure}[H]
  \centering
  \begin{bytefield}[endianness=big]{32}
    \bitheader{31,0} \\
    \wordbox{1}{ Program Counter } \\
    \begin{rightwordgroup}{Metric 1}
      \wordbox{1}{ Number of events, \#bytesaccording to size given in header }
    \end{rightwordgroup} \\
    \wordbox{2}{\vdots} \\
    \begin{rightwordgroup}{Metric N}
      \wordbox{1}{ Number of events, \#bytes according to size given in header }
    \end{rightwordgroup}
  \end{bytefield}
  \caption{Instruction Trace: Binary Format Body Element.}
  \label{fig:instr_trace_el}
\end{figure}
