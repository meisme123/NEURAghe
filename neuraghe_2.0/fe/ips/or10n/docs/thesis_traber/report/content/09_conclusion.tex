%%%%%%%%%%%%%%%%%%%%%%%%%%%%%%%%%%%%%%%%%%%%%%%%%%%%%%%%%%%%%%%%%%%%%%%
%%%%%%%%%%%%%%%%%%%%%%%%%%%%%%%%%%%%%%%%%%%%%%%%%%%%%%%%%%%%%%%%%%%%%%%
%%%%%                                                                 %
%%%%%     <file_name>.tex                                             %
%%%%%                                                                 %
%%%%% Author:      <author>                                           %
%%%%% Created:     <date>                                             %
%%%%% Description: <description>                                      %
%%%%%                                                                 %
%%%%%%%%%%%%%%%%%%%%%%%%%%%%%%%%%%%%%%%%%%%%%%%%%%%%%%%%%%%%%%%%%%%%%%%
%%%%%%%%%%%%%%%%%%%%%%%%%%%%%%%%%%%%%%%%%%%%%%%%%%%%%%%%%%%%%%%%%%%%%%%

\chapter{Conclusion}

\label{chapter:conclusion}

During this thesis the OpenRISC \gls{ISA} was significantly extended to support
vectorial instructions, a more powerful multiplier and bit counting
instructions. Similarly misaligned memory access
and interrupt capabilities were added to the \orion core. For more efficient
debugging in the future, debug facilities were added to the platform such that
it is possible to attach gdb to a running \orion core. By adding all those
features to \orion, we are now on a feature level that allows our core to
compete with commercial micro-controllers available on the market.

The instruction set extensions allowed us to achieve a performance gain of up to
5x on vectorial code compared to the base OpenRISC instruction set. When bit
counting operations can be employed, we were able to achieve a 35x higher
performance than before.
Our LLVM compiler is able to automatically generate code for the new vectorial
and MAC instructions and thus no modifications in the source code of existing
applications are necessary to take advantage of the extensions.

All our instruction set extensions added only $25\%$ area to \orion while the
pipeline stage delay was not affected by our modifications. If we look at the
\gls{PULP} cluster only $2\%$ of area was added due to the additional
instructions.

In terms of energy efficiency we could achieve an energy efficiency boost of up
to $67\%$ for specific applications, while on average $45\%$ energy could be
saved with our extensions.
